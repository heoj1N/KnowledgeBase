\section{Matrix Calculus and Gradients}
Differentiating functions with respect to vectors and matrices is crucial in backpropagation, 
gradient-based optimization, etc. Let’s outline some common rules:

\subsection{Gradient w.r.t.\ a Vector}
If $f(\mathbf{x})$ is a scalar function, and $\mathbf{x} \in \mathbb{R}^n$, 
the gradient $\nabla_{\mathbf{x}} f(\mathbf{x})$ is a vector in $\mathbb{R}^n$:
\[
\nabla_{\mathbf{x}} f(\mathbf{x})
= \begin{pmatrix}
\frac{\partial f}{\partial x_1} \\
\frac{\partial f}{\partial x_2} \\
\vdots \\
\frac{\partial f}{\partial x_n}
\end{pmatrix}.
\]
\textbf{Chain rule example:} If $\mathbf{y} = A \mathbf{x}$, then $f(\mathbf{x}) = g(\mathbf{y})$, 
we have $\nabla_{\mathbf{x}} f(\mathbf{x}) = A^\top \nabla_{\mathbf{y}} g(\mathbf{y}).$

\subsection{Gradient w.r.t.\ a Matrix}
If $F(A) \in \mathbb{R}$ for a matrix $A \in \mathbb{R}^{m \times n}$, 
the gradient is also a matrix of the same shape, where
\[
(\nabla_A F)_{ij} = \frac{\partial F}{\partial a_{ij}}.
\]
A common identity is:
\[
\nabla_A \, \text{tr}(BA) = B^\top,
\]
where $\text{tr}(\cdot)$ is the trace operator. 

\subsection{Backpropagation in Neural Networks}
All of the partial derivatives needed in neural networks (weights, biases, etc.) 
ultimately reduce to these matrix/vector calculus rules combined with the chain rule. 
\[
\frac{\partial \, \text{Loss}}{\partial W} 
= \frac{\partial \, \text{Loss}}{\partial \, \text{Output}} 
\cdot \frac{\partial \, \text{Output}}{\partial W}.
\]











